Cortar el calabacín y la zanahoria en tiras finas, como si fueran spaghettis. Para ello, puede usarse un cuchillo bien afilado o un cortador de verduras (tipo pela patatas).

Estos spaghettis de verdura deberán cocerse en agua hirviendo con sal, a fuego bajo, durante 3 o 5 minutos. Este tiempo es suficiente para que la verdura se cueza pero no se deshaga.

En una sartén, sofreír la cebolla con un poco de aceite de oliva. A continuación, añadir la carne salpimentada y dejar que se cocine a fuego lento. Finalmente, añadir el tomate bien picado y seguir cocinando durante unos minutos a fuego lento.

Llegados a este punto, los spaghetti están listos para ser servidos junto a su salsa.

Recomendaciones: usar tomates frescos, pimienta negra, orégano y una cucharada de azúcar en la salsa, y queso parmesano para finalizar.
